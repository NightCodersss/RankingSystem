\documentclass[12pt,a4paper]{scrartcl}
\usepackage[utf8]{inputenc}
\usepackage[english,russian]{babel}
\usepackage{indentfirst}
\usepackage{misccorr}
\usepackage{graphicx}
\usepackage{amsmath}
\begin{document}

\section{Система ранжирования}



\section{Зависимости}

\begin{itemize}
    \item C++14
    \item UbjsonCpp  - библиотека для работы протокола (repo: https://github.com/NightCodersss/UbjsonCpp)
    \item cppunit - библиотека для unit-тестирования (используется в Ubjson)
\end{itemize}

\section{Разработчику}
Помимно информации, изложенной здесь, следует изучить документацию, сгенерированную doxygen'ом (директории latex и html в корне проекта).

\subsection{Структура}
\label{subsec:generalStructure}
Структура выглядит так:
(Фронт-энд или другой сервис) \rightarrow Ranking Sytem \rightarraw IndexServer
Где '\rightarrow' обозначет подключение от клиена к серверу.

\section{Программистам-пользвателям}
См. структуру \ref{sec:generalStructure}

\section{Человеку, который запускает}
См. структуру \ref{sec:generalStructure}

\end{document}
