\documentclass[12pt,a4paper]{article}
\usepackage[utf8]{inputenc}
\usepackage[english,russian]{babel}
\usepackage{indentfirst}
\usepackage{misccorr}
\usepackage{graphicx}
\usepackage{amsmath}
\usepackage{mathtools}
\begin{document}

\section{Программисту-разработчику}

\subsection{Устройство хранилища}
Есть три вида хранилищ, каждому из которых соответствует класс:
	\begin{itemize}
		\item Глобальное $(BigStorage)$
		\item Среднего уровня $(KeyValueStorage)$
		\item Примитивное $(Storage)$
	\end{itemize}

Глобальное хранилище имеет для каждой пары (слово, TextID) по два хранилища среднего уровня -- прямое и обратное.
Хранилище среднего уровня представляет из себя абстракцию, позволяющую хранить пары ключ-значение в файле.
Для его реализации используется примитивное хранилище.

Оно представляет собой набор блоков и следующие операции над ними:
\begin{itemize}
	\item Добавить значение в блок с номером $n$
	\item Удалить значение из блока с номером $n$
	\item Получить значение из блока с номером $n$
	\item Получить итератор в порядке убывания
\end{itemize}
В этом хранилище поддерживается посортированный по убыванию порядок значений.

Хранилище среднего уровня использует эти операции для реализации более высокоуровневого хранилища, предоставляющего
операции сходные со структурой данных $map$. Это достигается засчет хранения диапазонов значений для каждого блока
и возможности искать по ним. Эта табличка хранится в отдельном файле.

В глобальном хранилище прямые и обратные хранилища -- индексы. В прямом мы храним пары (DocID, Rank). В обратном (Rank, DocID).
Они используются для организации добавления и удаления документов в индекс. Добавление реализуется через коммиты.

Структура коммита: кортежи (слово, TextID, DocID, Rank)

\end{document}
