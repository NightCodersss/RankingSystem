\section*{Зависимости }


\begin{DoxyItemize}
\item C++14
\item Ubjson\-Cpp -\/ библиотека для работы протокола (repo\-: \href{https://github.com/NightCodersss/UbjsonCpp}{\tt https\-://github.\-com/\-Night\-Codersss/\-Ubjson\-Cpp})
\item cppunit -\/ библиотека для unit-\/тестирования (используется в Ubjson)
\item json -\/ библиотека для чтения конфига (repo\-: \href{https://github.com/nlohmann/json}{\tt https\-://github.\-com/nlohmann/json})
\end{DoxyItemize}

\section*{Разработчику }

Помимо информации, изложенной здесь, следует изучить документацию, сгенерированную doxygen’ом (директории latex и html в корне проекта).

\subsection*{Структура \{\#subsec\-:general\-Structure\} }

Структура выглядит так\-: (Фронт-\/энд или другой сервис) \$\{Southern protocol\}\$ Ranking System \$\{Northern protocol\}\$ Index\-Server Где \$''\$ обозначет подключение от клиента к серверу.

\subsection*{Протокол }

В связи с тем, что Ubjson парсится на лету, не все поля обязательны. В силу гибкости протокола незначительное изменение или значительное добавление не должно требовать больших трудов. Что такое южный и северный легко понять отсюда \mbox{[}subsec\-:general\-Structure\mbox{]}.

\subsubsection*{Южный}

Запрос.

Необходимо передавать строчку query, которая хранит в себе запрос. Поле \$amount\$ не обязательно. \begin{DoxyVerb}{
    "query": "what is ...?"
    "amount": "1000000"
}
\end{DoxyVerb}


Ответ. \begin{DoxyVerb}{
    "docs": [ {...}, {...}, ],
    "amount": 12243,
}
\end{DoxyVerb}


Планируется добавить параметры, каким образом представлять информацию о документе обратно.

Заглушка работает так\-: слушает некоторый порт, по приходу данных, до переноса строки (\$ n\$), приходящие данные пакует в ubjson и отправляет системе ранжирования. По приходу данных от системы ранжирования (то есть системы поиска)(в ubjson) заглушка “их отвечает” в json (обычном).

\subsubsection*{Северный}

Запрос.

Такой же, как “южный”, но с параметром -\/ какой индекс использовать (по какому тексту), то есть\-: \begin{DoxyVerb}{
    "query": "what is ...?",
    "index_id": "aa1234df",
    "fields": ["docname", "author", "snippet"],
}
\end{DoxyVerb}


Ответ.

Такой же, как у южного.

\subsubsection*{Реализация}

Ясно, что нехорошо ждать каждого ответа. Глобально, есть два пути решения этой проблемы\-: использование асинхронных операций ввода-\/вывода (с callback’ами) и использование большого количества потоков. Существенных различий нет (нам известных), за исключением того, что код с большим количеством callback’ов плохо читаем и некрасив.

Итак, реализация такова, что на каждое соединение выделется отдельный поток, в котором соединение работает синхронно.

\subsection*{Конфигурация }

Следует обратить внимание, что конфигурация считывается в json, а не в ubjson. См. “конфигурация” для человека, который запускает.

\section*{Программистам-\/пользователям }

См. структуру \mbox{[}subsec\-:general\-Structure\mbox{]}

\section*{Человеку, который запускает }

См. структуру \mbox{[}subsec\-:general\-Structure\mbox{]}

\subsection*{Конфигурация }

Очень важно соблюдать типы занчений\-: строки должны быть в кавычках, числа без. Это не проверяется и приводит к повисанию (бага библиотеки работы с json). Пример конфига Ranking\-Server’а\-: \begin{DoxyVerb}{
    "texts": [
        {
            "servers": [{"host": "localhost", "port": "14000"}],
            "factor": {"person": 1.8, "article": 1.2},
            "name": "Title",
            "index_id": "1",
        },
        {
            "servers": [{"host": "", "port": "14000"}],
            "factor": 0.9,
            "name": "Useless information",
            "index_id": "100400ab",
        },
    ]
\end{DoxyVerb}
 