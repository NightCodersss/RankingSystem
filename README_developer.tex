\documentclass[12pt,a4paper]{article}
\usepackage[utf8]{inputenc}
\usepackage[english,russian]{babel}
\usepackage{indentfirst}
\usepackage{misccorr}
\usepackage{graphicx}
\usepackage{amsmath}
\usepackage{mathtools}
\begin{document}

\section{Система ранжирования}
\section{Разработчику}
Помимо информации, изложенной здесь, следует изучить документацию, сгенерированную doxygen'ом (директории latex и html в корне проекта).

\subsection{Сборка документации}
Команда:
\begin{verbatim}
make doc
\end{verbatim}
Соберет всю документацию (tex в pdf, md; doxygen) и сделает коммит только документации.




\subsubsection{Реализация}
Ясно, что нехорошо ждать каждого ответа. 
Глобально, есть два пути решения этой проблемы: использование асинхронных операций ввода-вывода (с callback'ами) и использование большого количества потоков. Существенных различий нет (нам известных), за исключением того, что код с большим количеством callback'ов плохо читаем и некрасив.

Итак, реализация такова, что на каждое соединение выделется отдельный поток, в котором соединение работает синхронно.

Следует обратить внимание, что конфигурация считывается в json, а не в ubjson.
\end{document}
