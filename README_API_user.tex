\documentclass[12pt,a4paper]{article}
\usepackage[utf8]{inputenc}
\usepackage[english,russian]{babel}
\usepackage{indentfirst}
\usepackage{misccorr}
\usepackage{graphicx}
\usepackage{amsmath}
\usepackage{mathtools}
\begin{document}

\section{Система ранжирования}
\section{Программистам-пользователям API}
Для начала следует прчитать документацию README\_deployer и README\_tuner.

\subsection{Протокол}
В связи с тем, что Ubjson парсится на лету, не все поля обязательны. 
В силу гибкости протокола незначительное изменение или значительное добавление не должно требовать больших трудов.
Что такое южный и северный легко понять отсюда \ref{subsec:generalStructure}.

\subsubsection{Южный}
Запрос.

Необходимо передавать строчку query, которая хранит в себе запрос.
Поле $amount$ не обязательно.
\begin{verbatim}
{
	"query": "what is ...?"
	"amount": "1000000"
}
\end{verbatim}

Ответ.

\begin{verbatim}
{
	"docs": [ {...}, {...}, ],
	"amount": 12243,
}
\end{verbatim}


Планируется добавить параметры, каким образом представлять информацию о документе обратно.

Заглушка работает так: слушает некоторый порт, по приходу данных, до переноса строки ($\backslash n$), приходящие данные пакует в ubjson и отправляет системе ранжирования. По приходу данных от системы ранжирования (то есть системы поиска)(в ubjson) заглушка "их отвечает" в json (обычном).

\subsubsection{Северный}
Запрос.

Такой же, как "южный", но с параметром - какой индекс использовать (по какому тексту), то есть:
\begin{verbatim}
{
	"query": "what is ...?",
	"index_id": "aa1234df",
	"fields": ["docname", "author", "snippet"],
}
\end{verbatim}

Ответ.

Такой же, как у южного. 
\end{document}
